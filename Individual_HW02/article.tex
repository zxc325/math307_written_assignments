\documentclass[11pt]{article}
\usepackage{setspace}
\setstretch{1}
\usepackage{amsmath,amssymb, amsthm}
\usepackage{graphicx}
\usepackage{bm}
\usepackage[hang, flushmargin]{footmisc}
\usepackage[colorlinks=true]{hyperref}
\usepackage[nameinlink]{cleveref}
\usepackage{footnotebackref}
\usepackage{url}
\usepackage{listings}
\usepackage[most]{tcolorbox}
\usepackage{inconsolata}
\usepackage[papersize={8.5in,11in}, margin=1in]{geometry}
\usepackage{float}
\usepackage{caption}
\usepackage{esint}
\usepackage{url}
\usepackage{enumitem}
\usepackage{subfig}
\usepackage{wasysym}
\newcommand{\ilc}{\texttt}
\usepackage{etoolbox}
\usepackage{algorithm}
\usepackage{changepage}
% \usepackage{algorithmic}
\usepackage[noend]{algpseudocode}
\usepackage{tikz}
\usetikzlibrary{matrix,positioning,arrows.meta,arrows}
\patchcmd{\thebibliography}{\section*{\refname}}{}{}{}
% \PassOptionsToPackage{hyphens}{url}\usepackage{hyperref}

\providecommand{\myceil}[1]{\left \lceil #1 \right \rceil }
\providecommand{\myfloor}[1]{\left \lfloor #1 \right \rfloor }


\begin{document}



\title{\textbf{MATH 307: Individual Homework 2}}


\author{
Shaochen (Henry) ZHONG, \ilc{sxz517@case.edu}}

\date{Due and submitted on 02/10/2021 \\ Spring 2021, Dr. Guo}
\maketitle




\subsection*{Problem 1}
\textit{Prove that $Q(x)$, the set of all polynomials with rational coefficients with the regular polynomial multiplication and addition is a ring.}\newline

\noindent For the simplicity of discussion, let's assume we have $f, g, k \in Q(x)$ with $f = a_0 + a_1 x + a_2 x^2 + ... + a_m x^m$, $g = b_0 + b_1 x + b_2 x^2 + ... + b_n x^n$, and $k = c_0 + c_1 x + c_2 x^2 + ... + c_j x^j$.

\noindent We have $(Q(x), +, 0)$ to be an Abelian group as:
\begin{itemize}
    \item It is a closure as for $f + g = a_0 + b_0 + a_1 x + b_1 x + ... + a_m x^m + b_n x^n$ is also a polynomial with rational coefficient and therefore also $\in Q(x)$.
    \item It shows associativity as for $f, g, k \in Q(x)$, $f + (g + k) = (f + g) + k$.
    \item It has the (additive) identity of $0$ for $f + 0 = f$.
    \item It has the inverse of $-f$ as $f + (-f) = 0$.
    \item It also shows commutativity with $f + g = g + f$.
\end{itemize}


\noindent On the other hand we have $(Q(x), \times, 1)$ to be a monoid as:

\begin{itemize}
    \item It is a closure as we have $f \cdot g = a_0 b_0 + a_1 b_1 x^2 + ... + a_m b_n x^{mn}$ to be a polynomial with rational coefficient and therefore also $\in Q(x)$.
    \item It shows associativity as for $f, g, k \in Q(x)$, $f \cdot (g \cdot k) = (f \cdot g) \cdot k$.
    \item It has the (multiplicative) identity of $1$ for $f \cdot 1 = f$.
\end{itemize}

Now to check the distributive property, for $f \cdot (g + h)$ we have $a_0 b_0 + a_1 b_1 x^2 + ... + a_m b_n x^{mn} + a_0 c_0 + a_1 c_1 x^2 + ... + a_m c_j x^{mj} = (f \cdot g) + (f \cdot h)$. So the distributive property is proven and $(Q(x), +, \times)$ is therefore a ring.



\subsection*{Problem 2}
\textit{Is $\mathbb{Z}$, the set of integers with the usual addition and multiplication, a field? Justify your answer.}\newline

$(\mathbb{Z}, +, \times)$ is not a field. First it is clear that $(\mathbb{Z}, +, 0)$ is supposed to be the Abelian group and $(\mathbb{Z}, \times, 1)$ is supposed to be the commutative monoid -- as we can't have a multiplicative inverse for every integers $\in \mathbb{Z}$. Which implies for $(\mathbb{Z}, +, \times)$ to be a field, it is required that every element in $\mathbb{Z}$ which is not the additive inverse ($0$) to have an inverse with respect to $\times$. But the only multiplicative inverse that are $\in \mathbb{Z}$ are $-1$ and $ 1$. Say if we have $2$, which is an integer that is not $0$, but we can't have its multiplicative inverse $\frac{1}{2}$ to be $\in \mathbb{Z}$. Thus, $(\mathbb{Z}, +, \times)$ is not a field.

\end{document}

