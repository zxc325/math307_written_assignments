\documentclass[11pt]{article}
\usepackage{setspace}
\setstretch{1}
\usepackage{amsmath,amssymb, amsthm}
\usepackage{graphicx}
\usepackage{bm}
\usepackage[hang, flushmargin]{footmisc}
\usepackage[colorlinks=true]{hyperref}
\usepackage[nameinlink]{cleveref}
\usepackage{footnotebackref}
\usepackage{url}
\usepackage{listings}
\usepackage[most]{tcolorbox}
\usepackage{inconsolata}
\usepackage[papersize={8.5in,11in}, margin=1in]{geometry}
\usepackage{float}
\usepackage{caption}
\usepackage{esint}
\usepackage{url}
\usepackage{enumitem}
\usepackage{subfig}
\usepackage{wasysym}
\newcommand{\ilc}{\texttt}
\usepackage{etoolbox}
\usepackage{algorithm}
\usepackage{changepage}
% \usepackage{algorithmic}
\usepackage[noend]{algpseudocode}
\usepackage{tikz}
\usetikzlibrary{matrix,positioning,arrows.meta,arrows}
\patchcmd{\thebibliography}{\section*{\refname}}{}{}{}
% \PassOptionsToPackage{hyphens}{url}\usepackage{hyperref}

\providecommand{\myceil}[1]{\left \lceil #1 \right \rceil }
\providecommand{\myfloor}[1]{\left \lfloor #1 \right \rfloor }


\begin{document}



\title{\textbf{MATH 307: Individual Homework 1}}


\author{
Shaochen (Henry) ZHONG, \ilc{sxz517@case.edu}}

\date{Due and submitted on 02/08/2021 \\ Spring 2021, Dr. Guo}
\maketitle



\section*{Section 2.4}
\subsection*{Problem 1}
\subsubsection*{(a)}
It is a monoid as it has the properties of \textit{closure} for each $a, b \in A$ we have $a + b \in A$, \textit{associativity} for $a, b, c \in A$ we have $a + (b + c) = (a + b) + c$, and \textit{identity} for $e = 0$.

\subsubsection*{(b)}
It is not a group as for any $a, b \in A$ and either of $a,b \neq 0$, we can't have $a + b = b + a = e = 0$.

\subsubsection*{(c)}
As we have checked above that $\{+, \mathbb{N}, 0\}$ is not a group, it is also not an Abelian group.

\subsection*{Problem 2}

$\{+, \mathbb{Z}, 0\}$ is an Abelian group as it has the properties of \textit{closure} for each $a, b \in A$ we have $a + b \in A$, \textit{associativity} for $a, b, c \in A$ we have $a + (b + c) = (a + b) + c$, and \textit{identity} for $e = 0$.

It also has \textit{inverse} for all $x \in A$ as $-x$, where $-x$ is also $\in A$ and $x + (-x) = 0 = e$. And it certainly shows \textit{commutativity} property with $a + b = b + a$ for all  $a, b \in A$.\newline

\noindent However, $\{\times, \mathbb{Z}, 1\}$ is only a monoid as it has the properties of \textit{closure} for each $a, b \in A$ we have $a \times b \in A$, \textit{associativity} for $a, b, c \in A$ we have $a \times (b \times c) = (a \times b) \times c$, and  \textit{identity} for $e = 1$. But
we can't say it is a group as we might have something $a = 2$ where the \textit{inverse} is $\frac{1}{2}$ which is $\not\in \mathbb{Z}$.


\end{document}

