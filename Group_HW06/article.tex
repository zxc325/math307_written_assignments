\documentclass[11pt]{article}
\usepackage{setspace}
\setstretch{1}
\usepackage{amsmath,amssymb, amsthm}
\usepackage{graphicx}
\usepackage{bm}
\usepackage[hang, flushmargin]{footmisc}
\usepackage[colorlinks=true]{hyperref}
\usepackage[nameinlink]{cleveref}
\usepackage{footnotebackref}
\usepackage{url}
\usepackage{listings}
\usepackage[most]{tcolorbox}
\usepackage{inconsolata}
\usepackage[papersize={8.5in,11in}, margin=1in]{geometry}
\usepackage{float}
\usepackage{caption}
\usepackage{esint}
\usepackage{url}
\usepackage{enumitem}
\usepackage{subfig}
\usepackage{wasysym}
\newcommand{\ilc}{\texttt}
\usepackage{etoolbox}
\usepackage{algorithm}
\usepackage{changepage}
% \usepackage{algorithmic}
\usepackage[noend]{algpseudocode}
\usepackage{tikz}
\usepackage{gensymb}
\usetikzlibrary{matrix,positioning,arrows.meta,arrows}
\patchcmd{\thebibliography}{\section*{\refname}}{}{}{}
% \PassOptionsToPackage{hyphens}{url}\usepackage{hyperref}

\providecommand{\myceil}[1]{\left \lceil #1 \right \rceil }
\providecommand{\myfloor}[1]{\left \lfloor #1 \right \rfloor }
\providecommand{\qbm}[1]{\begin{bmatrix} #1 \end{bmatrix}}
\providecommand{\qpm}[1]{\begin{pmatrix} #1 \end{pmatrix}}
\providecommand{\norm}[1]{\left\lVert #1 \right\rVert}
\providecommand{\len}[1]{\left| #1 \right|}

\begin{document}




\title{\textbf{MATH 307: Group Homework 6}}


\author{\textit{Group 8}\\
Shaochen (Henry) ZHONG, Zhitao (Robert) CHEN, John MAYS, Huaijin XIN\\ \ilc{\{sxz517, zxc325, jkm100, hxx200\}@case.edu}}

\date{Due and submitted on 03/19/2021 \\ Spring 2021, Dr. Guo}
\maketitle




\subsection*{Problem 1}
\textit{See HW instruction.}\newline

\subsection*{Problem 2}
\textit{See HW instruction.}\newline

\subsection*{Problem 3}
\textit{See HW instruction.}\newline

Known that $M_{ij}^* = \overline{M_{ji}}$, we have:

\begin{align*}
    LHS &= (AB)^*_{ij} = \overline{(AB)_{ji}} = \sum^{n}_{k} \overline{A_{jk} B_{ki}} \\
    RHS &= (B^* A^*)_{ij} = \sum^{n}_{k} B^{*}_{ik} A^{*}_{kj} = \sum^{n}_{k} \overline{B_{ki} A_{jk}} \\
    \Longrightarrow& \  (AB)^*_{ij} = (B^* A^*)_{ij}
\end{align*}

\subsection*{Problem 4}
\textit{See HW instruction.}\newline

\subsection*{Problem 5}
\textit{See HW instruction.}\newline


\end{document}

